\documentclass[11pt]{article}
\usepackage{latexsym}
\usepackage{amsmath,fullpage,amsthm,fancyhdr,amsfonts,amssymb}
\usepackage[svgnames]{xcolor}
\usepackage{graphicx}
\usepackage{float}
\usepackage{import, pst-plot}
\def\emptyline{\vspace{12pt}}

\usepackage{caption}
\usepackage{subcaption}



\pagestyle{fancy}
\lhead{ AMATH 574}
\chead{Hai Zhu}
\rhead{Student ID: 1323888}
\lfoot{}
\cfoot{\thepage}
\rfoot{}
\renewcommand{\headrulewidth}{0.1pt}
\renewcommand{\footrulewidth}{0.1pt}
\setlength{\textwidth}{6.2in}
\setlength{\textheight}{8.7in}
\setlength{\voffset}{-.7in}
\setlength{\headsep}{26pt}
\setlength{\parindent}{10pt}

\begin{document}

\title{\Large\bf Homework 2}
\author{Hai Zhu}
\date{due date: \today}
\maketitle
\thispagestyle{fancy}
\renewcommand{\qed}{\hfill \mbox{\raggedright \rule{0.07in}{0.1in}}} % for the use of end of proof mark

\begin{enumerate}
%-------------------------------------------------------------------------------------
%---Problem \#4.1 -------------------------------------------------------------------
    \item Problem \#4.1 
            
			\begin{enumerate}
			
				\item
					Determine the matrices $A^+$ and $A^-$, defined as
					\[
					A^+=R\Lambda^+R^{-1} \text{  and  } 
					A^-=R\Lambda^-R^{-1},
					\]
					for the acoustics equations
					\[
					\begin{bmatrix}
					p\\u
					\end{bmatrix}_t+
					\begin{bmatrix}
					u_0&K_0 \\ 1/\rho_0&u_0
					\end{bmatrix}
					\begin{bmatrix}
					p\\u
					\end{bmatrix}_x=0.
					\]
					
				\item	
					Determine the waves $W_{i-1/2}^1$ and $W_{i+1/2}^2$ that result from arbitrary data $Q_{i-1}$ and $Q_i$ for this system.			

			\end{enumerate}
		
		\vskip 5pt
        \noindent{\bf Solution}
        \vskip 5pt
        	
        	\begin{enumerate}
        	
				\item
					First, we need to diagonalize coefficient matrix $A=\begin{bmatrix}u_0&K_0\\1/\rho_0&u_0\end{bmatrix}$. Let $c_0=\sqrt{K_0/\rho_0}$. The following are eigenvalues and corresponding eigenvectors:
					\[
					\lambda^+=u_0+c_0 \leftrightarrow 
					r^+=\begin{bmatrix}\rho_0c_0\\1\end{bmatrix} \text{,   } 
					\lambda^-=u_0-c_0 \leftrightarrow 
					r^-=\begin{bmatrix}-\rho_0c_0\\1\end{bmatrix}
					\]  
					 	
					Thus under basis generated by $R=[r^-|r^+]$, matrix $A$ can be transformed into diagonal matrix
					\[ R^{-1}AR=\Lambda=\begin{bmatrix}u_0-c_0&\\&u_0+c_0\end{bmatrix}	\]						We take case $u_0=0$ for example:					
					\[
					\Rightarrow \Lambda^+=\begin{bmatrix}0&\\&c_0\end{bmatrix}
					\text{,  }
					\Lambda^-=\begin{bmatrix}-c_0&\\&0\end{bmatrix}
					\]
					
					Now we can compute $A^+$ and $A^-$. Let $z_0=\rho_0c_0$
					\begin{align*}
					A^+=R\Lambda^+R^{-1}
					&=\begin{bmatrix}-z_0&z_0\\1&1\end{bmatrix}
					\begin{bmatrix}0&\\&c_0\end{bmatrix}\frac{1}{2z_0}
					\begin{bmatrix}-1&z_0\\1&z_0\end{bmatrix}\\
					&=\begin{bmatrix}c_0/2&c_0z_0/2\\c_0/2z_0&c_0/2\end{bmatrix}
					\end{align*}
					\begin{align*}
					A^-=R\Lambda^-R^{-1}
					&=\begin{bmatrix}-z_0&z_0\\1&1\end{bmatrix}
					\begin{bmatrix}-c_0&\\&0\end{bmatrix}\frac{1}{2z_0}
					\begin{bmatrix}-1&z_0\\1&z_0\end{bmatrix}\\
					&=\begin{bmatrix}-c_0/2&c_0z_0/2\\c_0/2z_0&-c_0/2\end{bmatrix}
					\end{align*}
					
				\item
					To compute $W_{i-1/2}^1$ and $W_{i-1/2}^2$, it is simply finding the expression of $\Delta Q_i$ under basis $R$ from (a).
					\begin{align*}
					Q_i-Q_{i-1} &= \alpha_{i-1/2}^-r^- + \alpha_{i-1/2}^+r^+ 
					=R\begin{bmatrix}\alpha_{i-1/2}^-\\\alpha_{i-1/2}^+\end{bmatrix}\\
					&= W_{i-1/2}^1 + W_{i-1/2}^2
					\end{align*}
					
					Here we assume $Q=[p, u]^T$. Then do left multiplication of the above expression by $R^{-1}$ on both sides.
					\begin{align*}
					\Rightarrow 
					\begin{bmatrix}\alpha_{i-1/2}^-\\\alpha_{i-1/2}^+\end{bmatrix}&=
					\frac{1}{2z_0}\begin{bmatrix}-1&z_0\\1&z_0\end{bmatrix}
					\begin{bmatrix}p_i-p_{i-1}\\u_i-u_{i-1}\end{bmatrix}\\
					&=\frac{1}{2z_0}
					\begin{bmatrix}
					-p_i+p_{i-1}+z_0u_i-z_0u_{i-1}\\
					p_i-p_{i-1}+z_0u_i-z_0u_{i-1}
					\end{bmatrix}
					\end{align*} 
					
					And hence
					\begin{align*}
					&W_{i-1/2}^1=\alpha_{i-1/2}^-r^-
					=\frac{1}{2z_0}(-p_i+p_{i-1}+z_0u_i-z_0u_{i-1})
					\begin{bmatrix}-z_0\\1\end{bmatrix}\\
					&W_{i-1/2}^2=\alpha_{i-1/2}^+r^+
					=\frac{1}{2z_0}(p_i-p_{i-1}+z_0u_i-z_0u_{i-1})
					\begin{bmatrix}z_0\\1\end{bmatrix}
					\end{align*}
        	
        	\end{enumerate} 
\qed
\newpage  
%-------------------------------------------------------------------------------------        
%---Problem \#4.2 --------------------------------------------------------------------
    \item Problem \#4.2 
    
			If we apply upwind method to advection equation $q_t+\bar{u}q_x=0$ with $\bar{u}>0$, and choose the time step so that $\bar{u}\Delta t=\Delta x$, then the method reduces to 
			$
			Q_i^{n+1}=Q_{i-1}^n
			$
			The initial data simply shifts one grid cell each time step and the exact solution is obtained, up to the accuracy of the initial data. This is a nice property for a numerical method to have and is sometimes called the unit CFL condition
			
			\begin{enumerate}
			
				\item Sketch figures to illustrate the wave-propagation interpretation of this result.
				\item Does the Lax-Friedrichs method satisfy the unit CFL condition? Does the two-step Lax-Wendroff method?
				\item Show that the exact solution is also obtained for the constant-coefficient acoustics equations with $u_0=0$ if we choose the time step so that $c\Delta t=\Delta x$ and apply Godunov's method. Determine the formulas for $p_i^{n+1}$ and $u_i^{n+1}$ that result in this case, and show how they are related to the solution obtained from characteristic theory.
				\item Is it possible to obtain a similar exact result by a suitable choice of $\Delta t$ in the case where $u_0\neq 0$ in acoustics?			
			
			\end{enumerate}      
        \noindent{\bf Solution}
        	\begin{enumerate}
				\item
					The following are figures of wave propagation interpretation when $\bar{u}\Delta t=\Delta x$,
					\begin{figure}[H]
						\centering
						\begin{subfigure}{.5\textwidth}
  							\centering
  							\includegraphics[width=0.85\linewidth,height=4.1in]{problem_4_2_figure_a.png}
  							\caption{$\bar{u}>0$}
  							\label{fig:sub1}
						\end{subfigure}%
						\begin{subfigure}{.5\textwidth}
  							\centering
  							\includegraphics[width=0.85\linewidth,height=4.1in]{problem_4_2_figure_b.png}
  							\caption{$\bar{u}<0$}
  							\label{fig:sub2}
						\end{subfigure}
						\caption{A figure with two subfigures}
					\label{fig:test}
					\end{figure}
					
				\item
					For advection equation, Lax-Friedrichs method has the following form:
					\[
					Q_i^{n+1}=\frac{1}{2}(Q_{i-1}^n+Q_{i+1}^n)
					-\frac{\Delta t}{2\Delta x}(\bar{u}Q_{i+1}^n-\bar{u}Q_{i-1}^n)
					\]				
					If we choose $\bar{u}\Delta t=\Delta x$, then
					\begin{align*}
					Q_i^{n+1}&=\frac{1}{2}(Q_{i-1}^n+Q_{i+1}^n-Q_{i+1}^n+Q_{i-1}^n)\\
					&=Q_{i-1}^n
					\end{align*}
					which satisfies unit CFL condition.\\
					
					Two-step Lax-Wendroff method can be expressed in for following form for advection equation
					\begin{align*}
					Q_i^{n+1}= Q_i^n-\frac{\Delta t}{\Delta x}&(F_{i+1/2}^n-F_{i-1/2}^n)\\
					=Q_i^n-\frac{\Delta t}{\Delta x}
					&(\bar{u}Q_{i+1/2}^{n+1/2}-\bar{u}Q_{i-1/2}^{n+1/2}) \\
					=Q_i^n-\frac{\Delta t}{\Delta x}&\{\bar{u}[ \frac{1}{2}(Q_{i+1}^n+Q_i^n)-\frac{\Delta t}{2\Delta x}(\bar{u}Q_{i+1}^n-\bar{u}Q_i^n)]\\
					&-\bar{u}[\frac{1}{2}(Q_{i-1}^n+Q_i^n)-\frac{\Delta t}{2\Delta x}(\bar{u}Q_i^n-\bar{u}Q_{i-1})]\}
					\end{align*}
					If we choose $\bar{u}\Delta t=\Delta x$, then
					\begin{align*}
					Q_i^{n+1}&=Q_i^n-\frac{1}{2}(Q_{i+1}^n+Q_i^n)+\frac{1}{2}(Q_{i+1}^n-Q_i^n)+\frac{1}{2}(Q_{i-1}^n+Q_i^n)-\frac{1}{2}(Q_i^n-Q_{i-1}^n)\\
					&=Q_{i-1}^n
					\end{align*}
					which again satisfies unit CFL condition.
					
				\item
					Let's find out the formula of Godunov's method for this constant-coefficient acoustic system when $c\Delta t=\Delta x$, and then discuss whether this formula will give us exact solution in that case. Here we are going to use the relation derived in problem 4.1 (b).
					\[
					Q_i^{n+1}=\begin{bmatrix}p_i^{n+1}\\u_i^{n+1}\end{bmatrix}
					=\begin{bmatrix}p_i^n\\u_i^n\end{bmatrix}-\frac{\Delta t}{\Delta x}
					[\lambda^1 W_{i+1/2}^1 + \lambda^2W_{i-1/2}^2]
					\]
					Recall that we have determine the relation between $\alpha$ and components of $Q$ in problem 4.1 (b).
					\begin{align*}
					\begin{bmatrix}p_i^{n+1}\\u_i^{n+1}\end{bmatrix} = 
					\begin{bmatrix}p_i^n\\u_i^n\end{bmatrix}-\frac{\Delta t}{\Delta x}
					&[\lambda^1 \alpha_{i+1/2}^1 r^1 + \lambda^2\alpha_{i-1/2}^nr^2]\\
					=\begin{bmatrix}p_i^n\\u_i^n\end{bmatrix}-\frac{\Delta t}{\Delta x}
					&\{-\frac{c_0}{2z_0}(-p_{i+1}^n+p_i^n+z_0u_{i+1}^n-z_0u_i^n)
					\begin{bmatrix}-z\\1\end{bmatrix} \\
					& +\frac{c_0}{2z_0}(p_i^n-p_{i-1}^n+z_0u_i^n-zu_{i-1}^n)
					\begin{bmatrix}z\\1\end{bmatrix}\}\\
					=\begin{bmatrix}p_i^n\\u_i^n\end{bmatrix}-\frac{\Delta t}{\Delta x}
					&\frac{c_0}{2z_0}\begin{bmatrix}2z_0p_i^n-z_0(p_{i+1}^n+p_{i-1}^n)+z_0^2(u_{i+1}^n-u_{i-1}^n)\\
					2z_0u_i^n+(p_{i+1}^n-p_{i-1}^n-z_0(u_{i+1}^n+u_{i-1}^n))
					\end{bmatrix}
					\end{align*}
					If we choose $c\Delta t=\Delta x$, then
					\[
					\begin{bmatrix}p_i^{n+1}\\u_i^{n+1}\end{bmatrix} = \begin{bmatrix}
					\frac{1}{2}(p_{i+1}^n+p_{i-1}^n)-\frac{z_0}{2}(u_{i+1}^n-u_{i-1}^n)\\
					-\frac{1}{2z_0}(p_{i+1}^n-p_{i-1}^n)+\frac{1}{2}(u_{i+1}^n+u_{i-1}^n)
					\end{bmatrix}
					\]
					
					Now let's consider how they are related to solution obtained from characteristic theory. Initial data inside each cell splits into two waves with wave speeds $\pm c_0$. For cell $i-1$, $i$, and $i+1$, when $c\Delta t=\Delta x$, data inside $i$ has moved exactly one cell after $\Delta t$. So what's inside cell $i$ is nothing but information generated from cell $i-1$ and $i+1$. It is actually the right moving wave from cell $i-1$, and left moving wave from cell $i+1$.
					\begin{align*}
					Q_{i-1,r}^n &= \frac{1}{2z_0}(p_{i-1}^n+z_0u_{i-1}^n)
					\begin{bmatrix}z_0\\1\end{bmatrix}\\
					Q_{i+1,l}^n &= \frac{1}{2z_0}(-p_{i+1}^n+z_0u_{i+1}^n)
					\begin{bmatrix}-z_0\\1\end{bmatrix}
					\end{align*}
				
					\begin{align*}
					Q_i^{n+1} &=Q_{i-1,r}^n+Q_{i+1,l}^n\\
					 &=\begin{bmatrix}
					\frac{1}{2}(p_{i+1}^n+p_{i-1}^n)-\frac{z_0}{2}(u_{i+1}^n-u_{i-1}^n)\\
					-\frac{1}{2z_0}(p_{i+1}^n-p_{i-1}^n)+\frac{1}{2}(u_{i+1}^n+u_{i-1}^n)
					\end{bmatrix}
					\end{align*}
					Thus exact solution is also obtained.
				
				\item
					For the case when $u_0\neq 0$, we have two eigenvalues with different absolute value, which means the left moving wave has a different speed from right moving wave. It is unlikely both left moving and right moving initial data would move exactly some multiple of cell distance.\\
					
					However, with a bit of luck, $\lambda^1 = u_0+c_0$ might be some multiple of $\lambda^2 = u_0-c_0$. In that case, when left moving wave moves exactly one cell, the right moving wave will move a few cells. We are able to utilize this property to come up with some numerical method that would give us exact solution. But that kind of numerical solution is not likely to be used in practice, because it violates CFL condition. For the fast moving wave, we should require it move less or equal than a cell after one time step, which makes the slow moving wave not able to move exactly one cell in this case. So we are not be able to come up with some practical method and choose some time step to obtain a similar exact solution.
				        	
        	
        	\end{enumerate}

\qed
\newpage
%-------------------------------------------------------------------------------------
%---Problem \#6.1 --------------------------------------------------------------------
    \item Problem \#6.1 \\
            Verify the following results from integrating the piecewise linear solution $\tilde{q}^n(x,t_{n+1})$
            \begin{align*}
            Q_i^{n+1}&=\frac{\bar{u}\Delta t}{\Delta x}(Q_{i-1}^n+\frac{1}{2}(\Delta x-\bar{u}\Delta t)\sigma_{i-1}^n)+(1-\frac{\bar{u}\Delta t}{\Delta x})(Q_i^n-\frac{1}{2}\bar{u}\Delta t \sigma_i^n)\\
            &=Q_i^n-\frac{\bar{u}\Delta t}{\Delta x}(Q_i^n-Q_{i-1}^n)-\frac{1}{2}\frac{\bar{u}\Delta t}{\Delta x}(\Delta x -\bar{u}\Delta t)(\sigma_i^n-\sigma_{i-1}^n)
            \end{align*}

        \vskip 5pt
        \noindent{\bf Solution}
        \vskip 5pt
			
			Here we assume that $\bar{u}>0$. The cell average inside cell $i$ comes from two parts. One is the flux moving inside from left end point. And the other is what's left inside cell $i$ after some flux moving outside from right end point. Let's compute these two effects separately, and then combine those two effects and get the formula.
			\begin{align*}
			F_{i-1/2}^n &= \int_{x_{i-1/2}-\bar{u}\Delta t}^{x_{i-1/2}}
			[Q_{i-1}^n+\sigma_{i-1}^n(x-x_{i-1})]dx \\
			&= \bar{u}\Delta t Q_{i-1}^n + \frac{1}{2}\sigma_{i-1}^n
			(x-x_{i-1})^2|_{x_{i-1/2}-\bar{u}\Delta t}^{x_{i-1/2}}\\
			&=\bar{u}\Delta t[Q_{i-1}^n+\frac{1}{2}(\Delta x-\bar{u}\Delta t)\sigma_{i-1}^n]
			\end{align*}
			\begin{align*}
			Q_{left}^n &= \int_{x_{i-1/2}}^{x_{i+1/2}-\bar{u}\Delta t}
			[Q_i^n+\sigma_i^n(x-x_i)]dx \\
			&= (\Delta x - \bar{u}\Delta t) Q_i^n + \frac{1}{2}\sigma_i^n
			(x-x_i)^2|_{x_{i-1/2}}^{x_{i+1/2}-\bar{u}\Delta t}\\
			&=(\Delta x - \bar{u}\Delta t) Q_i^n-\frac{1}{2}\bar{u}\Delta t\sigma_i^n
			(\Delta x-\bar{u}\Delta t)
			\end{align*}
            
            Thus we can get
            \begin{align*}
            &Q_i^{n+1} \Delta x= F_{i-1/2}^n + Q_{left}^n\\
            \Rightarrow &
            Q_i^{n+1}=Q_i^n-\frac{\bar{u}\Delta t}{\Delta x}(Q_i^n-Q_{i-1}^n)-\frac{1}{2}\frac{\bar{u}\Delta t}{\Delta x}(\Delta x -\bar{u}\Delta t)(\sigma_i^n-\sigma_{i-1}^n)
			\end{align*}             
\qed
\newpage
%-------------------------------------------------------------------------------------
%---Problem \#6.2 --------------------------------------------------------------------
    \item Problem \#6.2 \\
            Compute the total variation of the functions
            \begin{enumerate}
            	\item
            		\[
            		q(x)=\begin{cases}1 & \text{if } x<0,\\ \sin{\pi x} & \text{if } 0\leq x\leq3, \\2 & \text{if } x>3,\end{cases}
            		\]
            	\item
            		\[
            		q(x)=\begin{cases}1 &\text{if }x<0 \text{ or } x=3,\\1 &\text{if }0\leq x\leq 1\text{ or }2\leq x<3,\\-1 &\text{if } 1<x<2,\\2 &\text{if } x>3.\end{cases}
            		\]
            \end{enumerate}

        \vskip 5pt
        \noindent{\bf Solution}
        \vskip 5pt
				
			\begin{enumerate}
			
				\item
					For arbitrary function $q(x)$, total variation is defined as 
					\[
					TV(q)=\sup\sum_{j=1}^N|q(\xi_j)-q(\xi_{j-1})|,
					\]
					where the supremum is taken over all possible subdivision of real line.\\
					
					Firstly, consider the following subdivision of real line. And compute the variation.
					\[
					-\infty=\xi_0<0<\frac{1}{2}<\frac{3}{2}<\frac{5}{2}<3<\xi_N=\infty
					\]
					Then we have
					\begin{align*}
					TV(q)\geq &\sum_{j=1}^N |q(\xi_j)-q(\xi_{j-1})|\\
					= &|q(0)-q(\infty)|+|q(1/2)-q(0)|+|q(3/2)-q(1/2)|\\
					&+|q(5/2)-q(3/2)|+|q(3)-q(5/2)|+|q(\infty)-q(3)|\\
					= &1+1+2+2+1+2\\
					=&9
					\end{align*}
					
					If we can prove any subdivision is less or equal than $9$. Then we are done.\\
					
					For any subdivision 
					\[
					-\infty=\eta_0 < \eta_1<\cdots<\eta_{\tilde{N}}=\infty
					\]
					Combine this subdivision with the subdivision through which we get variation $9$	
					\[
					-\infty=\xi_0<0<\frac{1}{2}<\frac{3}{2}<\frac{5}{2}<3<\xi_N=\infty
					\]	
					Then we get a fine subdivision of real line, which can be represented as following:
					\[
					-\infty=\zeta_0<\cdots<0<\cdots<\zeta_i<\cdots<1/2<\cdots<3/2<\cdots<5/2<\cdots<3<\cdots<\zeta_{N'}=\infty
					\]	
					Since $\{\zeta_i\}$ is a fine subdivision, the variation we get should not decrease compare to variation we get from $\{\eta_j\}$, $\sum_{i=1}^{N'}|q(\zeta_i)-q(\zeta_i-1)|\geq \sum_{j=1}^{\tilde{N}}|q(\eta_j)-q(\eta_{j-1})|$.\\
						
					We can divide $\sum_{i=1}^{N'}|q(\zeta_i)-q(\zeta_i-1)|$ into sums over interval $(\xi_{j-1},\xi_{j})$. Then inside each interval $q(\zeta_i)-q(\zeta_i-1)$	 has the same sign. Then we can get rid of absolute value, and get
					\[
					\sum_{i=1}^{N'}|q(\zeta_i)-q(\zeta_i-1)| 
					= \sum_{k=1}^N|q(\xi_k)-q(\xi_{k-1})|
					\]
					
				\item
					We can use the same approach as in (a).\\
					
					Firstly, consider the following subdivision 
					\[
					-\infty =\xi_0<1<3/2<5/2<<\xi_N=\infty
					\]	
					Compute the variation of $q$ under this subdivision.
					\begin{align*}
					&\sum_{j=1}^{N}|q(\xi_j)-q(\xi_{j-1})|\\
					=&|q(1)-q(-\infty)|+|q(3/2)-q(1)|\\
					&+|q(5/2)-q(3/2)|+|q(\infty)-q(5/2)|\\
					=&0+2+2+1\\
					=&5
					\end{align*}
					
					Following the same argument as in (a), we can get the conclusion that any other subdivision won't increase variation of this function. Actually, this function is more obvious to see. Since it is piecewise constant, adding more points won't increase variation of this function.
			
			\end{enumerate}            

\qed
\newpage
%-------------------------------------------------------------------------------------
%---Problem \#6.3 --------------------------------------------------------------------
    \item Problem \#6.3 \\
            Show that any TVD method is monotonicity-preserving.

        \vskip 5pt
        \noindent{\bf Solution}
        \vskip 5pt
        
            Assume $q^n(x)$ approaches constant states as $x\rightarrow \pm\infty$.\\
            Then we should expect $q^{n+1}(x)$ approaches the same constant states as $x\rightarrow \pm\infty$.\\
            
            Let's consider the exact solution of linear system first ($q(x,t_n)$, $q(x,t_{n+1})$). For each point $x$ we care about at time $t_{n+1}$, it is easy to find out the domain of dependence at time $t_n$. Since the value at this point $x$ is nothing but linear combination of data from domain of dependence, and the values inside this domain of dependence are nearly the same, we should get a barely changed value at $x$ (The waves splitting into from $x$ at time $t_n$ and the waves combining at $(x,t_{n+1})$ are nearly the same). We can get $q(x,t_{n+1})$ goes to the same constant states as $q(x,t_n)$. As for numerical solutions, they should be a good approximation of exact solution. So we should expect they go to the same constant states. (Another way to say it is that we should have some restrictions on the coefficients of numerical methods, so that when we plug in the same value for each variable on the right hand side to update solution, the output should be the same value.)\\
            
            That kind of behavior is also what we should expect from cell averages.\\
            
            We can draw the conclusion that any TVD method is monotonicity-preserving by contradiction. WLOG, let's take decreasing case. And assume $Q^n_i\rightarrow a$ as $i\rightarrow +\infty$ and $Q^n_i\rightarrow b$ as $i\rightarrow -\infty$. If 
            \[
            Q_i^{n+1}\geq Q_{i+1}^{n+1} \text{  not for all } i,
            \]
            then the cell average function would have locally maximun and locally minimun.\\
            Then let's take a look at time $t_{n+1}$. Assume $Q^{n+1}_j$ is the smallest locally minimum, and $Q^{n+1}_k$ is the nearest locally maximum. Here we also assume $Q^{n+1}_j$ and $Q^{n+1}_k$ lie between $a$ and $b$. Because otherwise we can simply take the one goes beyond $[b,a]$, compute the variation, and get contradiction.
            \begin{align*}
            TV(Q^{n+1})\geq & |Q^{n+1}_{-\infty}-Q^{n+1}_j|+|Q^{n+1}_j-Q^{n+1}_k|+|Q^{n+1}_k-Q^{n+1}_{\infty}|\\
            = &Q^{n+1}_{-\infty}-Q^{n+1}_j+|Q^{n+1}_j-Q^{n+1}_k|+Q^{n+1}_k-Q^{n+1}_{\infty}\\
            = &a-b + Q^{n+1}_k-Q^{n+1}_j+|Q^{n+1}_j-Q^{n+1}_k|\\
            = &a-b + 2|Q^{n+1}_j-Q^{n+1}_k|\\
            > &a-b
            \end{align*}
            The last inequality is strict, because $j$, $k$ correspond to neighbourhood locally minimum and maximum. Here we get contradiction, this method is not TVD.\\
            
            Thus the assumption that this method is not monotonicity-preserving is wrong, which means any TVD method is monotonicity-preserving.
        
\qed
\newpage
%-------------------------------------------------------------------------------------
%---Problem \#6.4 --------------------------------------------------------------------
    \item Problem \#6.4 \\
            Show that 
            \[
            TV(Q^{n+1})\leq TV(\tilde{q}^n(\cdot,t_{n+1}))
            \]
            is valid by showing that, for any function $q(x)$, if we define discrete values $Q_i$ by averaging $q(x)$ over grid cells, then $TV(Q)\leq TV(q)$

        \vskip 5pt
        \noindent{\bf Solution}
        \vskip 5pt
        
        	Write down total variation formula for $Q$ and $q$ first.\\
        	\[
        	TV(Q)=\sum_{i=-\infty}^{\infty}|Q_i-Q_{i-1}|
        	\] 
        	\[
        	TV(q)=\sup\sum_{j=1}^N|q(\xi_j)-q(\xi_{j-1})|
        	\]
        	We might still assume $Q(\pm\infty)$ are constants. And the tails are small.\\
        	
        	We evolve $\tilde{q}^n(\cdot,t_n)$ to get $\tilde{q}(\cdot,t_{n+1})$. And then average this result to get $Q^{n+1}_i$.
        	\begin{align*}
        	&\int_{x_{i-1/2}}^{x_{i+1/2}}\tilde{q}^n(\xi,t_{n+1})d\xi=Q_i^{n+1}\\
        	\Rightarrow &\tilde{q}^n(x_{i,min},t_{n+1})\leq Q_i^{n+1}\leq\tilde{q}^n(x_{i,max},t_{n+1})
        	\end{align*}
        	First thing we can do is to get rid of the tails of $TV(Q)$, and get finite sum and finite number of cells.\\
        	
        	Second thing we can do is to find subdivision of this finite interval, so that we can compare the variation of $Q$ with variation of $q$. The chosen criterion is as follows
        	\begin{align*}
        	&\text{for local maximum } Q_i, \text{ choose } x_{i,max}\\
        	&\text{for local minimun } Q_j, \text{ choose } x_{j,min}\\
        	&\text{for other cells   } Q_k, \text{ choose any point as wish } 
        	\end{align*} 
        	As we can see, what do matter in the sum of $TV(Q)$ are those local maximum and local minimum. We can simplify the sum into terms that only contains these local maximum and minimum cells, and won't change the value of sum. By the chosen criterion, the corresponding value $\tilde{q}^{n}(\cdot,t_{n+1})$ will also be bigger or equal than the local maximum, or less or equal than the local minimum, which would make the variation sum of $\tilde{q}^n$ no less than variation of $Q^{n+1}$ on this finite interval. $TV(\tilde{q}^n(\cdot,t_{n+1}))\geq TV(Q^{n+1})-\epsilon$ for any $\epsilon$. Thus we can get the inequality.
        	\[
            TV(Q^{n+1})\leq TV(\tilde{q}^n(\cdot,t_{n+1}))
            \]
            
            
\qed
\newpage

%-------------------------------------------------------------------------------------
%---Problem \#6.5 --------------------------------------------------------------------
    \item Problem \#6.5 \\
           Show that the minmod slope guarantees that 
           \[
           TV(\tilde{q}^n(\cdot,t_n))\leq TV(Q^n)
           \] 
           will be satisfied in general, and hence the minmod method is TVD

        \vskip 5pt
        \noindent{\bf Solution}
        \vskip 5pt
        	WLOG, we only need to discuss different cases for three adjacent cells, since minmod slop of cell $i$ only has connection with $Q_{i-1}$, $Q_{i+1}$. If we can conclude that inside cell $i$ ( taking all these subsets will cover all cells), the total variation of $q$ will always less than total variation of $Q$. And then this argument can be proven.\\      	
        	
        	Different cases of three adjacent cells can be divided into two general cases.
        	\begin{align*}
        	&Q_i^n \in [Q_{i-1}^n, Q_{i+1}^n], \text{ assume $Q_{i-1}^n<Q_{i+1}^n$ }\\
        	&Q_i^n \in [Q_{i-1}^n, Q_{i+1}^n]^c
        	\end{align*}      	
			\begin{figure}[H]
						\centering
						\begin{subfigure}{1.0\textwidth}
  							\centering
  							\includegraphics[width=0.8\linewidth,height=2.2in]{problem_6_5.png}
  							\caption{case 1}
  							\label{fig:sub1}
						\end{subfigure}%
						\\
						\begin{subfigure}{1.0\textwidth}
  							\centering
  							\includegraphics[width=0.8\linewidth,height=2.2in]{problem_6_5b.png}
  							\caption{case 2}
  							\label{fig:sub2}
						\end{subfigure}
					\label{fig:test}
			\end{figure}        	
        	
        	We may only need to deal with $Q_i^n \in [Q_{i-1}^n, Q_{i+1}^n]$, because for the other case minmod slope would give $0$ inside cell $i$. And $\tilde{q}^n$ in this cell will be constant $Q^n_i$. Then no matter which point we choose in cell $i$, it won't make more contribution to the variation of $\tilde{q}^n$ than the contribution $Q^n_i$ makes to the variation of $Q^n$. \\
        	
        	What we are going to discuss is the case $Q_i^n \in [Q_{i-1}^n, Q_{i+1}^n]$, and the interval $[x_{i-1/2},x_i]$. The length of this interval is $\Delta x/2$. And the slope is
        	\[
        	\sigma_i^n=minmod(\frac{Q_i^n-Q^n_{i-1}}{\Delta x},\frac{Q^n_{i+1}-Q_i^n}{\Delta x})
        	\]
        	Thus for the linear function inside $[x_{i-1/2},x_i]$, $\tilde{q}^n(x_{i-1/2},t_n)$ would be the smallest (assume $Q_{i-1}^n<Q_{i+1}^n$).
        	\begin{align*}
        	\tilde{q}^n(x_{i-1/2},t_n) &= Q_i^n - \frac{\Delta x}{2} \sigma_i^n\\
        	&\geq Q_i^n - \frac{\Delta x}{2} \frac{Q_i^n-Q^n_{i-1}}{\Delta x}\\
        	&\geq \frac{1}{2}(Q_i^n+Q_{i-1}^n)
        	\end{align*}
        	Similarly, we can get on interval $[x_{i},x_{i+1/2}]$,$\tilde{q}^n(x_{i+1/2},t_n)$ would be the biggest, and
        	\[
        	\tilde{q}^n(x_{i+1/2},t_n)\leq \frac{1}{2}(Q_i^n+Q_{i+1}^n)
        	\]
        	Combine these two, we can get the total variation of $\tilde{q}^n$ on cell $i$
        	\[
        	TV(\tilde{q}^n)|_{[x_{i-1/2},x_{i+1/2}]}\leq \frac{1}{2}(Q_{i+1}^n-Q_i^n)+\frac{1}{2}(Q_i^n-Q_{i-1}^n)
        	\]
        	Here we don't want to combine $-Q_i^n$ and $Q_i^n$, in order to understand this inequality better. Both terms on the right hand side are parts of the total variation of $Q^n$. Actually, we only use half of variation contributed from cell $i$ and its adjacent cells. The other half will be used when we do the same evaluation on cell $i-1$ and $i+1$. Or they might never be used depending on whether $Q_{i-1}^n \in [Q_{i-2}^n,Q_i^n]$ or not.\\
        	
        	Here we can see a pattern. The total variation of $\tilde{q}^n$ will never exceed total variation of $Q^n$, based on what we have analyzed on each cell.\\
        	
        	Thus we can conclude now
        	\[
            TV(\tilde{q}^n(\cdot,t_n))\leq TV(Q^n)
            \] 
%            \hfil\includegraphics[width=4.0in]{problem_3_5.png}\hfil
            
            
\qed
\newpage
%-------------------------------------------------------------------------------------
%---Problem \#6.6 --------------------------------------------------------------------
    \item Problem \#6.6 \\
			Show that taking 
			\[
			\delta_{i-1/2}^n=Q_i^n-Q_{i-1}^n
			\]
        	in
        	\[
        	F_{i-1/2}^n=\bar{u}^-Q_i^n+\bar{u}^+Q_{i-1}^n+\frac{1}{2}|\bar{u}|(1-|\frac{\bar{u}\Delta t}{\Delta x}|)\delta_{i-1/2}^n
        	\]
        	corresponds to using the downwind slope for $\sigma$ in both cases $\bar{u}>0$ and $\bar{u}<0$, and that the resulting flux gives the Lax-Wendroff method.
        
        \vskip 5pt
        \noindent{\bf Solution}
        \vskip 5pt
        	
        	For $\bar{u}>0$,
        	\begin{align*}
        		F_{i-1/2}^n&=\bar{u}Q_{i-1}^n+\frac{1}{2}\bar{u}(1-\frac{\bar{u}\Delta t}{\Delta x})\delta_{i-1/2}^n \\
        		&=\bar{u}Q_{i-1}^n+\frac{1}{2}\bar{u}(1-\frac{\bar{u}\Delta t}{\Delta x})(Q_i^n-Q_{i-1}^n)\\
        		&=\frac{1}{2}\bar{u}(Q_{i-1}^n+Q_i^n)-\frac{1}{2}\frac{\bar{u}^2\Delta t}{\Delta x}(Q_i^n-Q_{i-1}^n)
        	\end{align*}
        	This flux results in Lax-Wendroff method, which can be verified by plugging $F_{i-1/2}^n$ into flux form
        	\[
        	Q_i^{n+1}=Q_i^n-\frac{\Delta t}{\Delta x}(F_{i+1/2}^n-F_{i-1/2}^n)
        	\]
            
            For $\bar{u}<0$,
        	\begin{align*}
        	F_{i-1/2}^n&=\bar{u}Q_{i}^n-\frac{1}{2}\bar{u}(1+\frac{\bar{u}\Delta t}{\Delta x})\delta_{i-1/2}^n \\
        	&=\bar{u}Q_{i}^n-\frac{1}{2}\bar{u}(1+\frac{\bar{u}\Delta t}{\Delta x})(Q_i^n-Q_{i-1}^n)\\
        	&=\frac{1}{2}\bar{u}(Q_{i-1}^n+Q_i^n)-\frac{1}{2}\frac{\bar{u}^2\Delta t}{\Delta x}(Q_i^n-Q_{i-1}^n)
        	\end{align*}
        	This is the same as the flux for case $\bar{u}>0$. And it results in Lax-Wendroff method, which can be verified by plugging $F_{i-1/2}^n$ into flux form
        	\[
        	Q_i^{n+1}=Q_i^n-\frac{\Delta t}{\Delta x}(F_{i+1/2}^n-F_{i-1/2}^n)
        	\]
        	
\qed
\newpage
%-------------------------------------------------------------------------------------
%---Problem Additional-----------------------------------------------------------------
    \item Problem Additional \\
			Show that the flux-limiter method 
			\begin{align*}
			\bar{u}>0, \ \ Q_i^{n+1} = &Q_i^n-v(Q_i^n-Q_{i-1}^n)\\
			&-\frac{1}{2}v(1-v)[\phi(\theta_{i+1/2}^n)(Q_{i+1}^n-Q_i^n)-\phi(\theta_{i-1/2}^n)(Q_i^n-Q_{i-1}^n)]\\
			\bar{u}<0, \ \ Q_i^{n+1} = &Q_i^n-v(Q_{i+1}^n-Q_{i}^n)\\
			&+\frac{1}{2}v(1+v)[\phi(\theta_{i+1/2}^n)(Q_{i+1}^n-Q_i^n)-\phi(\theta_{i-1/2}^n)(Q_i^n-Q_{i-1}^n)]
			\end{align*}
			can be written as a wave limiter method as:
			\[
			Q_i^{n+1}=Q_i^n-\frac{\Delta t}{\Delta x}(\bar{u}^+W_{i-1/2}+\bar{u}^-W_{i+1/2})-\frac{\Delta t}{\Delta x}(\tilde{F}_{i+1/2}-\tilde{F}_{i-1/2}),
			\]
			where $W_{i-1/2}=Q_i^n-Q_{i-1}^n$ and the "correction flux" is
			\[
			\tilde{F}_{i-1/2}=\frac{1}{2}|\bar{u}|(1-\frac{\Delta t}{\Delta x}|\bar{u}|)\tilde{W}_{i-1/2},
			\]
			with the limited waves $\tilde{W}$ defined by
			\[
			\tilde{W}_{i-1/2}=\phi(\theta_{i-1/2})W_{i-1/2}.
			\]
			The ratio $\theta_{i-1/2}$ is defined in (6.35) and the function $\phi$ might be one of limiters from (6.39).
			
        \vskip 5pt
        \noindent{\bf Solution}
        \vskip 5pt
        
        	Some algebra to verify that these two forms are equal.\\
        	
        	For $\bar{u}>0$, wave-limiter form can be rewritten as 
        	\begin{align*}
        	Q_i^{n+1} = &Q_i^n-v(Q_i^n-Q_{i-1}^n)\\
			&-\frac{\Delta t}{\Delta x}[\frac{1}{2}\bar{u}(1-v)\phi(\theta_{i+1/2})(Q_{i+1}^n-Q_i^n)]-\frac{1}{2}\bar{u}(1-v)\phi(\theta_{i-1/2})(Q_i^n-Q_{i-1}^n)\\
			= &Q_i^n-v(Q_i^n-Q_{i-1}^n)\\
			&-\frac{1}{2}v(1-v)[\phi(\theta_{i+1/2}^n)(Q_{i+1}^n-Q_i^n)-\phi(\theta_{i-1/2}^n)(Q_i^n-Q_{i-1}^n)],
        	\end{align*}
            which is the same as flux-limiter method.\\
            
            For $\bar{u}<0$, wave-limiter form can be rewritten as 
        	\begin{align*}
        	Q_i^{n+1} = &Q_i^n-v(Q_{i+1}^n-Q_{i}^n)\\
			&-\frac{\Delta t}{\Delta x}[-\frac{1}{2}\bar{u}(1+v)\phi(\theta_{i+1/2})(Q_{i+1}^n-Q_{i}^n)+\frac{1}{2}\bar{u}(1+v)\phi(\theta_{i-1/2})(Q_{i}^n-Q_{i-1}^n)]\\
			= &Q_i^n-v(Q_{i+1}^n-Q_{i}^n)\\
			&+\frac{1}{2}v(1+v)[\phi(\theta_{i+1/2}^n)(Q_{i+1}^n-Q_i^n)-\phi(\theta_{i-1/2}^n)(Q_i^n-Q_{i-1}^n)],
        	\end{align*}
            which is the same as flux-limiter method.
\qed           

\end{enumerate}

\end{document}