\documentclass{article}
\usepackage{fullpage}
%%%%%%%%%%%%%%%%%%%%%%%%%%%%%%%%%%%
\usepackage{graphicx}
\usepackage{epstopdf}
\usepackage{amsthm}
\usepackage{amsmath}
\usepackage{amssymb}
\usepackage{caption}
\usepackage{subcaption}
\usepackage[all]{xy}



\everymath{\displaystyle}

\newenvironment{solution}
  {\begin{proof}[Solution]}
  {\end{proof}}
\newtheorem{definition}{Definition}
\newtheorem{example}{Example}
\newtheorem{theorem}{Theorem}
\newtheorem{remark}{Remark}
\newtheorem{lemma}{Lemma}
\newtheorem{inequality}{Inequality}
\newtheorem{proposition}{Proposition}

\providecommand{\abs}[1]{\left|#1\right|}
\providecommand{\norm}[1]{\lVert#1\rVert}
%%%%%%%%%%%%%%%%%%%%%%%%%%%%%%%%%%%
\begin{document}
\title{AMATH574 Conservation Laws and Finite Volume Methods\\ Winter Quarter 2015\\ Project Draft: F-wave method for nonlinear equations with spatially varying fluxes.}
\author{Instructor : Professor Randall Leveque\\ Student: Hai Zhu(zhuh1106@uw.edu), Xin Yang (yangxin@uw.edu)\\ Due date: Wednesday, Feb. 18, 2015}
%\date
\maketitle

\section{Abstract}
We study the f-wave method for elastic waves in heterogeneous media.
\section{Introduction and Overview}
\subsection{Physical Model}
In this project, we are considering 1D elasticity equations
\begin{align}
\epsilon(x,t)_t-u(x,t)_x=0 \label{strain}\\
\rho(x) u(x,t)_t-\sigma(\epsilon,x)_x=0 \label{n2}
\end{align}
where, $\epsilon$ is the strain, $u$ is the velocity, $\rho$ is the density and $\sigma$ is the stress. These are classical mechanics equations in Lagrangian form. Equation \eqref{strain} comes from the definition of strain. If $v(x)$ is the displacement of small element at $x$, then the strain, i.e. the relative change of displacement is $\epsilon=\frac{dv(x)}{dx}$. Equation \eqref{strain} is then obtained by taking a time derivative. The stress is the force per unit area. Therefore Equation \eqref{n2} is simply Newton's second law. In order to close the system, often the constitutive law $\sigma=\sigma(\epsilon)$ is introduced. The relation depends on the material, for instance:
\begin{itemize}
\item The simplest linear media:
\begin{align}
\sigma(x,t)=K(x)\epsilon(x,t)
\label{lins}
\end{align}
here $K(x)$ is the bulk modulus or Young's modulus which determines the stiffness of the material.
\item Relatively simple nonlinear model with quadratic relation:
\begin{align}
\sigma(x,t)=K(x)\epsilon(x,t)+\beta K^2(x)\epsilon^2
\label{quads}
\end{align}
\item Nonlinear model with quadratic relation:
\begin{align}
\sigma(x,t)=\exp(K(x)\epsilon(x,t))-1
\label{exps}
\end{align}
\item (Periodically) Layered media:
\begin{align}
K(x)=\left\{
\begin{array}{cc}
K_A & \mbox{if }j\delta<x<(j+\alpha)\delta \mbox{ for some integer } j,\\
K_B & \mbox{otherwise.}
\end{array}
\right.
\end{align}
Here $\delta$ is the period and $\alpha$ is the proportion of the first type of medium.
\end{itemize}
In small $\epsilon$ case, Equation \eqref{exps} is approximated by Equation \eqref{quads} if $\beta=0.5$ while they are both approximated by the linear model in Equation \eqref{lins}.
\subsection{Issues with variable coefficients equations and the idea of f-wave method}
As we have seen in class, the variable coefficients can sometimes be treated as the solution to an additional PDE. For instance, if we assume the linear media for the 1D elasticity equations above. $K(x)_t=0$ could be added to the system. Therefore,
\begin{align*}
\epsilon(x,t)_t-u(x,t)_x=0 \\
\rho(x) u(x,t)_t-(K(x)\epsilon)_x=0 \\
K(x)_t=0
\end{align*}
becomes a nonlinear (linearly degenerate) constant coefficient system of conservation law. This additional equation introduces a new line of characteristics $x=x_{i-1/2}$ which is stationary. As a result of this new discontinuity locating on the edges of cells, if one still work with the original system with one less equation, the decomposition of $Q_i-Q_{i-1}$ using eigenvectors (w-waves) would be wrong without the consideration of the jump across $x_{i-1/2}$. However, notice that in the wave-propagation form updating formula
\begin{align}
Q_i^{n+1}=Q_i^{n}-\frac{\Delta t}{\Delta x}\left[\sum (\lambda^p)^-W_{i+1/2}^p+\sum (\lambda^p)^+ W_{i-1/2}^p\right]
\end{align}
there's no contribution of the wave with speed zero. Therefore, one may seek for different approaches. As can be seen from the name, the idea of f-wave method is to decompose the flux difference $f(Q_i)-f(Q_{i-1})$ using eigenvector of the original system:
\begin{align}
f(Q_i)-f(Q_{i-1})=\sum \beta^p_{i-1/2} r^p_{i-1/2}\equiv \sum Z^p_{i-1/2}.
\end{align}
Since the Rankine-Hugoniot condition across the discontinuity with speed zero means the continuity of the flux, both the wave-propagation formula and the Rankine-Hugoniot condition suggest that considering propagations of discontinuities of the fluxes is more appropriate.
\subsection{Implementation}
The f-wave method uses the updating formula:
\begin{align}
Q_i^{n+1}=Q_i^{n}-\frac{\Delta t}{\Delta x}\left[\sum sign((\lambda^p)^-) Z_{i+1/2}^p+\sum sign((\lambda^p)^+) Z_{i-1/2}^p\right]
\end{align}
Therefore, the change from w-wave to f-wave is simple provided the approximate Riemann solver is given. Hence, we give the following procedure in the construction of the codes for the Riemann solver:
\begin{enumerate}
\item Obtain the approximate Jacobian $A_{i-1/2}$ or, alternatively, get the eigenvalues $s^p$ and eigenvectors $r^p$ of the approximate Jacobian.
\item Decompose the flux difference to get $\beta=R^{-1}(f(Q_i)-f(Q_{i-1}))$. The p-th f-wave is then given by $\beta^p r^p$.
\item Sum up $A^- \Delta q$ all the left-going f-wave, and $A^+ \Delta q$ all the right-going wave. In this problem, since we only have two equations and the sound speed is $\pm_c$, $A^- \Delta q$ contains the 1-f-wave and $A^+ \Delta q$ contains the 2-f-wave.
\item If needed, $W^p=Z^p/s^p$ the w-waves can be recovered by a division. However, numerical error may occur when $s^p$ is close to zero.
\end{enumerate}
The f-wave approach has already been implemented in CLAWPACK. One need only return the waves as fwaves in the Riemann solver and set the flag\\
\verb clawdata.use_fwaves = \verb True  in \verb setrun.py .
\section{Objective}
\begin{enumerate}
\item Implement the f-wave method for nonlinear elastic waves in heterogeneous media. Reproduce some of the figures in \cite{bale2002} \cite{leveque2003} and \cite{ketcheson2012}.----Almost done.
\item Have a more comprehensive discussion of the f-wave method and problem. Mostly on the verifications of the details of the method mentioned in the paper and textbook. p. 314 $\&$ p. 333 in the textbook. Some aspects include:
    \begin{itemize}
    \item The disadvantages of using cell-edge flux functions in wave-propagation algorithm metioned in \cite[p. 957]{bale2002}
    \item Justification of the Riemann solver used in \cite[p. 967]{bale2002}  ---done
    \item The non-conservation when using w-wave in wave-propagation algorithm for 1) nonlinear autonomous systems with simple Riemann solver (HLL? p. 328 in the textbook), 2) non-autonomous systems with "Roe average" Rieman solver.    ----partially done
%    \item The non-conservation when using w-wave in wave-propagation algorithm and possible fix approximate Riemann solvers for 1) nonlinear autonomous systems with simple Riemann solver (HLL? p. 328 in the textbook), 2) non-autonomous systems with "Roe average" Rieman solver. p. 318 in the textbook
    \item The slight difference of using $\mathcal{Z}=s\mathcal{W}$ in the limiter to get high-resolution methods. \cite[p. 964]{bale2002} p. 335 in the textbook.
    \item The new line of discontinuities caused by the discontinuities of the coefficient. \cite[p. 960]{bale2002}   ---done
    \item The breakdown of f-wave method in the case of singular flux (delta distribution).  \cite[p. 961]{bale2002}
    \end{itemize}
\item If possible, have some discussion on the dispersive properties of layered media i.e, solitons and shocks.
\end{enumerate}
The importance of the goals is decreasing in the order.
\section{Theoretical Background}
\subsection{Approximate Riemann Solver}
Since we are dealing with a nonlinear system, exact solutions to the Riemann problem is usually costly, the implementation of an approximate Riemann solver is inevitable, however, it is often enough to give good approximations as we shall see later in the numerical simulations. Here we use the approximate Riemann solver introduced in \cite{leveque2003}. Instead of giving the approximate Jacobian $A_{i-1/2}$ directly, we specify the eigenvalues and eigenvectors of $A_{i-1/2}$:
\begin{align}
r^1_{i-1/2}=r^1_{i-1}=\left[
                        \begin{array}{c}
                          1 \\ Z_{i-1} \\
                        \end{array}
                      \right], & \,\,\, s^1_{i-1/2}=-\sqrt{\frac{\sigma'_{i-1}(\epsilon_{i-1})}{\rho_{i-1}}}\\
r^2_{i-1/2}=r^2_{i}=\left[
                        \begin{array}{c}
                          1 \\ -Z_{i} \\
                        \end{array}
                      \right], &\,\,\, s^2_{i-1/2}=-\sqrt{\frac{\sigma'_{i}(\epsilon_{i})}{\rho_{i}}}
\end{align}
Here we are using the solution to the eigenvalue problem of the exact Jacobian but evaluating the eigenvalues and eigenvectors according to the cells where they are. This follows from the fact that there can be no transonic rarefaction waves for these equations. One issue about using this approximate Riemann solver in the standard wave propagation form is that the "Roe condition":
\begin{align}
A_{i-1/2}(Q_i-Q_{i-1})=f_{i}(Q_i)-f_{i-1}(Q_{i-1})
\end{align}
is not satisfied.
A quick verification can be seen by checking whether $Z^1_{i-1/2}=\beta^1 r^1=s^1_{i-1/2} \alpha^1r^1=s^1_{i-1/2}W^1_{i-1/2}$. After decomposition,
\[
\beta^1_{i-1/2}=\frac{\sigma_i-\sigma_{i-1}+Z_i(u_i-u_{i-1})}{Z_i+Z_{i-1}}
\]
while
\[
\alpha^1_{i-1/2}=\frac{\rho_i u_i-\rho_{i-1}u_{i-1}+Z_i(\epsilon_i-\epsilon_{i-1})}{Z_i+Z_{i-1}}
\]
with $s^1_{i-1/2}=-\sqrt{\frac{\sigma'_{i-1}(\epsilon_{i-1})}{\rho_{i-1}}}$
and $Z_i=\sqrt{\sigma'_{i}(\epsilon_{i})\rho_{i}}$. Calculation can be done to show that $ \beta^1 \neq s^1_{i-1/2} \alpha^1$. The direct consequence with the breakdown of the "Roe condition" is that the standard w-wave propagation scheme may not be conservative any more. However, it is straightforward to see that in the f-wave method, since we are decomposing $f_{i}(Q_i)-f_{i-1}(Q_{i-1})$ into waves, the flux difference will not change and be equal to the fluctuations which assembles the waves. As a result, the f-wave method allows the schemes to be conservative even if the "Roe condition" is not satisfied and it provides us more flexibility in choosing approximate Riemann solvers.
\subsection{Discretization of Flux Function}
	For spatially varying flux function, flux function $f(q,x)$ can be discretized with respect to $x$ in some manner consistent with a finite-volume interpretation. In this project, we are considering cell-centered flux functions $f_i(q)$ which holds throughout the $i$th cell. For sufficiently smooth $f$, this flux function might be defined simply by $f_i(q)=f(q,x_i)$. When cell-centered flux functions are used, the generalized Riemann problem at cell interface $x_{i-1/2}$ consists of the equation
	\[
	q_t + F_{i-1/2}(q,x)_x=0
	\]
	together with the initial data, where
	\[
	F_{i-1/2}(q,x)=\begin{cases}f_{i-1}(q) & \text{if } x<x_{i-1/2}\\
	f_i(q) & \text{if } x>x_{i-1/2}\end{cases}
	\]
	It is also natural to instead assume that $f_{i-1/2}(q)$ is specified at each cell interface. This cell-edge flux functions measure the flow between cell $i-1$ and cell $i$ which is used in flux-differencing algorithm. However, when we include high-resolution corrections, and apply flux limiter, it can be interpreted as applying the limiter functions to the waves propagation form resulting from the Riemann solutions.\\
	
	\noindent The wave propagation form for cell-edge flux functions can be related to the cell-centered flux approach by viewing the flux $f_{i-1/2}(q)$ as holding over the interval $[x_{i-1},x_i]$. But that would cause discontinuity in flux at the cell centers $x_i$. The wave propagation algorithm is based on solving Riemann problem at each interface of discontinuity. For our Riemann problem at $x_{i-1/2}$, we need to solve equations $q_t+f_{i-1/2}(q)_x=0$. Also there is a jump of flux at each cell center $x_i$, which requires to consider a second set of Riemann problems. Nontrivial waves can arise from these points because of the jump in flux. And as a result, we need to include flow generated from these discontinuities.
\subsection{Breakdown of F-wave Method}
The primary advantage of the f-wave method is that the entire flux difference is decomposed and carried by these propagating f-waves. Thus the flux is continuous across $x_{i-1/2}$. However, this is true only for conservation laws that have bounded solutions for which the flux is continuous everywhere. Details to be added.
	
\section{Computational Results}
See Figure \ref{travelw} for the soliton-like "travelling waves".
\begin{figure}
  % Requires \usepackage{graphicx}
  \includegraphics[width=0.5\textwidth]{frame0004fig1.png}
  \includegraphics[width=0.5\textwidth]{frame0008fig1.png}
  \includegraphics[width=0.5\textwidth]{frame0012fig1.png}
  \includegraphics[width=0.5\textwidth]{frame0020fig1.png}
  \includegraphics[width=0.5\textwidth]{frame0030fig1.png}
  \includegraphics[width=0.5\textwidth]{frame0040fig1.png}
  \caption{The decomposition of the initial single bump into soliton-like waves.}
  \label{travelw}
\end{figure}


\section{Summary and Conclusions}



%\section{Appendix A: MATLAB functions used and brief implementation explanation}
%\section{Appendix B: MATLAB codes}
%\section{Appendix C(optional)} Any algebraically intense calculations.

\begin{thebibliography}{9}
\bibitem{bale2002}
D. S. Bale, R. J. LeVeque, S. Mitran, and J. A. Rossmanith, SIAM J. Sci. Comput 24 (2002), 955-978.
\bibitem{leveque2002}
Finite Volume Methods for Nonlinear Elasticity in Heterogeneous Media
by R. J. LeVeque, Int. J. Numer. Meth. Fluids 40 (2002), pp. 93-104.
\bibitem{leveque2003}
Randall J. LeVeque and Darryl H. Yong, SIAM J. Appl. Math., 63 (2003), pp. 1539-1560.
\bibitem{ketcheson2012}
David I Ketcheson, Randall J. LeVeque Comm. Math. Sci. 10 (2012), pp. 859-874.

\end{thebibliography}

\end{document}
